\documentclass[12pt,fleqn,openany,letter]{book}
\usepackage{appendix}
\usepackage{minted}
%%%%%%%%%%%%%%%%%%%%%%%%%%%%%%%%%%%%%%%%%
% The Legrand Orange Book
% Structural Definitions File
% Version 2.0 (9/2/15)
%
% Original author:
% Mathias Legrand (legrand.mathias@gmail.com) with modifications by:
% Vel (vel@latextemplates.com)
%
% This file has been downloaded from:
% http://www.LaTeXTemplates.com
%
% License:
% CC BY-NC-SA 3.0 (http://creativecommons.org/licenses/by-nc-sa/3.0/)
%
%%%%%%%%%%%%%%%%%%%%%%%%%%%%%%%%%%%%%%%%%

%----------------------------------------------------------------------------------------
%	VARIOUS REQUIRED PACKAGES AND CONFIGURATIONS
%----------------------------------------------------------------------------------------

\usepackage{graphicx} % Required for including pictures
\graphicspath{{imagenes/}} % Specifies the directory where pictures are stored

\usepackage{tikz} % Required for drawing custom shapes

%\usepackage[spanish]{babel} % Spanish language/hyphenation
\usepackage[spanish, es-tabla]{babel} % Spanish language/hyphenation

\usepackage{enumitem} % Customize lists
\setlist{nolistsep} % Reduce spacing between bullet points and numbered lists

\usepackage{booktabs} % Required for nicer horizontal rules in tables

%\usepackage[table]{xcolor}
\usepackage{xcolor} % Required for specifying colors by name
%\definecolor{ocre}{RGB}{243,102,25} % Define the orange color used for highlighting throughout the book
\definecolor{ocre}{RGB}{80,97,123} % Define the orange color used for highlighting throughout the book


%----------------------------------------------------------------------------------------
%	MARGINS
%----------------------------------------------------------------------------------------



%\usepackage[top=3cm,bottom=3cm,left=3cm,right=3cm,headsep=12pt,a4paper]{geometry} % Page margins

\usepackage{geometry} % Required for adjusting page dimensions and margins

\geometry{
%	paper=a4paper, % Paper size, change to letterpaper for US letter size
	paper=letterpaper, % Paper size, change to letterpaper for US letter size
	top=3cm, % Top margin
	bottom=3cm, % Bottom margin
	left=3cm, % Left margin
	right=3cm, % Right margin
	headheight=13pt, % Header height
	footskip=1.4cm, % Space from the bottom margin to the baseline of the footer
	headsep=10pt, % Space from the top margin to the baseline of the header
	%showframe, % Uncomment to show how the type block is set on the page
}


%----------------------------------------------------------------------------------------
%	FONTS
%----------------------------------------------------------------------------------------

\usepackage{avant} % Use the Avantgarde font for headings
%\usepackage{times} % Use the Times font for headings
\usepackage{mathptmx} % Use the Adobe Times Roman as the default text font together with math symbols from the Sym­bol, Chancery and Com­puter Modern fonts

\usepackage{microtype} % Slightly tweak font spacing for aesthetics
\usepackage[utf8]{inputenc} % Required for including letters with accents
\usepackage[T1]{fontenc} % Use 8-bit encoding that has 256 glyphs


%----------------------------------------------------------------------------------------
%	BIBLIOGRAPHY AND INDEX
%----------------------------------------------------------------------------------------
\usepackage[babel]{csquotes}

\usepackage[backend=biber,style=apa]{biblatex}

%\usepackage[backend=biber,style=alphabetic,]{biblatex}

\DeclareLanguageMapping{spanish}{spanish-apa}

\addbibresource{bibliografia.bib} % BibTeX bibliography file
\defbibheading{bibempty}{}

%%%%

\usepackage{calc} % For simpler calculation - used for spacing the index letter headings correctly
\usepackage{makeidx} % Required to make an index
\makeindex % Tells LaTeX to create the files required for indexing

%----------------------------------------------------------------------------------------
%	MAIN TABLE OF CONTENTS
%----------------------------------------------------------------------------------------

\usepackage{titletoc} % Required for manipulating the table of contents

\contentsmargin{0cm} % Removes the default margin

% Part text styling
\titlecontents{part}[0cm]
{\addvspace{20pt}\centering\large\bfseries}
{}
{}
{}

% Chapter text styling
\titlecontents{chapter}[1.25cm] % Indentation
{\addvspace{12pt}\large\sffamily\bfseries} % Spacing and font options for chapters
{\color{ocre!60}\contentslabel[\Large\thecontentslabel]{1.25cm}\color{ocre}} % Chapter number
{\color{ocre}}
{\color{ocre!60}\normalsize\;\titlerule*[.5pc]{.}\;\thecontentspage} % Page number

% Section text styling
\titlecontents{section}[1.25cm] % Indentation
{\addvspace{3pt}\sffamily\bfseries} % Spacing and font options for sections
{\contentslabel[\thecontentslabel]{1.25cm}} % Section number
{}
{\hfill\color{black}\thecontentspage} % Page number
[]

% Subsection text styling
\titlecontents{subsection}[1.25cm] % Indentation
{\addvspace{1pt}\sffamily\small} % Spacing and font options for subsections
{\contentslabel[\thecontentslabel]{1.25cm}} % Subsection number
{}
{\ \titlerule*[.5pc]{.}\;\thecontentspage} % Page number
[]

% List of figures
\titlecontents{figure}[0em]
{\addvspace{-5pt}\sffamily}
{\thecontentslabel\hspace*{1em}}
{}
{\ \titlerule*[.5pc]{.}\;\thecontentspage}
[]

% List of tables
\titlecontents{table}[0em]
{\addvspace{-5pt}\sffamily}
{\thecontentslabel\hspace*{1em}}
{}
{\ \titlerule*[.5pc]{.}\;\thecontentspage}
[]

%----------------------------------------------------------------------------------------
%	MINI TABLE OF CONTENTS IN PART HEADS
%----------------------------------------------------------------------------------------

% Chapter text styling
\titlecontents{lchapter}[0em] % Indenting
{\addvspace{15pt}\large\sffamily\bfseries} % Spacing and font options for chapters
{\color{ocre}\contentslabel[\Large\thecontentslabel]{1.25cm}\color{ocre}} % Chapter number
{}
{\color{ocre}\normalsize\sffamily\bfseries\;\titlerule*[.5pc]{.}\;\thecontentspage} % Page number

% Section text styling
\titlecontents{lsection}[0em] % Indenting
{\sffamily\small} % Spacing and font options for sections
{\contentslabel[\thecontentslabel]{1.25cm}} % Section number
{}
{}

% Subsection text styling
\titlecontents{lsubsection}[.5em] % Indentation
{\normalfont\footnotesize\sffamily} % Font settings
{}
{}
{}

%----------------------------------------------------------------------------------------
%	PAGE HEADERS
%----------------------------------------------------------------------------------------

\usepackage{fancyhdr} % Required for header and footer configuration

\pagestyle{fancy}
\renewcommand{\chaptermark}[1]{\markboth{\sffamily\normalsize\bfseries\chaptername\ \thechapter.\ #1}{}} % Chapter text font settings
\renewcommand{\sectionmark}[1]{\markright{\sffamily\normalsize\thesection\hspace{5pt}#1}{}} % Section text font settings
\fancyhf{} \fancyhead[LE,RO]{\sffamily\normalsize\thepage} % Font setting for the page number in the header
\fancyhead[LO]{\rightmark} % Print the nearest section name on the left side of odd pages
\fancyhead[RE]{\leftmark} % Print the current chapter name on the right side of even pages
\renewcommand{\headrulewidth}{0.5pt} % Width of the rule under the header
\addtolength{\headheight}{2.5pt} % Increase the spacing around the header slightly
\renewcommand{\footrulewidth}{0pt} % Removes the rule in the footer
\fancypagestyle{plain}{\fancyhead{}\renewcommand{\headrulewidth}{0pt}} % Style for when a plain pagestyle is specified

% Removes the header from odd empty pages at the end of chapters
\makeatletter
\renewcommand{\cleardoublepage}{
\clearpage\ifodd\c@page\else
\hbox{}
\vspace*{\fill}
\thispagestyle{empty}
\newpage
\fi}

%----------------------------------------------------------------------------------------
%	THEOREM STYLES
%----------------------------------------------------------------------------------------

\usepackage{amsmath,amsfonts,amssymb,amsthm} % For math equations, theorems, symbols, etc

\newcommand{\intoo}[2]{\mathopen{]}#1\,;#2\mathclose{[}}
\newcommand{\ud}{\mathop{\mathrm{{}d}}\mathopen{}}
\newcommand{\intff}[2]{\mathopen{[}#1\,;#2\mathclose{]}}
\newtheorem{notation}{Notation}[chapter]

% Boxed/framed environments
\newtheoremstyle{ocrenumbox}% % Theorem style name
{0pt}% Space above
{0pt}% Space below
{\normalfont}% % Body font
{}% Indent amount
{\small\bf\sffamily\color{ocre}}% % Theorem head font
{\;}% Punctuation after theorem head
{0.25em}% Space after theorem head
{\small\sffamily\color{ocre}\thmname{#1}\nobreakspace\thmnumber{\@ifnotempty{#1}{}\@upn{#2}}% Theorem text (e.g. Theorem 2.1)
\thmnote{\nobreakspace\the\thm@notefont\sffamily\bfseries\color{black}---\nobreakspace#3.}} % Optional theorem note
\renewcommand{\qedsymbol}{$\blacksquare$}% Optional qed square

\newtheoremstyle{blacknumex}% Theorem style name
{5pt}% Space above
{5pt}% Space below
{\normalfont}% Body font
{} % Indent amount
{\small\bf\sffamily}% Theorem head font
{\;}% Punctuation after theorem head
{0.25em}% Space after theorem head
{\small\sffamily{\tiny\ensuremath{\blacksquare}}\nobreakspace\thmname{#1}\nobreakspace\thmnumber{\@ifnotempty{#1}{}\@upn{#2}}% Theorem text (e.g. Theorem 2.1)
\thmnote{\nobreakspace\the\thm@notefont\sffamily\bfseries---\nobreakspace#3.}}% Optional theorem note

\newtheoremstyle{blacknumbox} % Theorem style name
{0pt}% Space above
{0pt}% Space below
{\normalfont}% Body font
{}% Indent amount
{\small\bf\sffamily}% Theorem head font
{\;}% Punctuation after theorem head
{0.25em}% Space after theorem head
{\small\sffamily\thmname{#1}\nobreakspace\thmnumber{\@ifnotempty{#1}{}\@upn{#2}}% Theorem text (e.g. Theorem 2.1)
\thmnote{\nobreakspace\the\thm@notefont\sffamily\bfseries---\nobreakspace#3.}}% Optional theorem note

% Non-boxed/non-framed environments
\newtheoremstyle{ocrenum}% % Theorem style name
{5pt}% Space above
{5pt}% Space below
{\normalfont}% % Body font
{}% Indent amount
{\small\bf\sffamily\color{ocre}}% % Theorem head font
{\;}% Punctuation after theorem head
{0.25em}% Space after theorem head
{\small\sffamily\color{ocre}\thmname{#1}\nobreakspace\thmnumber{\@ifnotempty{#1}{}\@upn{#2}}% Theorem text (e.g. Theorem 2.1)
\thmnote{\nobreakspace\the\thm@notefont\sffamily\bfseries\color{black}---\nobreakspace#3.}} % Optional theorem note
\renewcommand{\qedsymbol}{$\blacksquare$}% Optional qed square
\makeatother

% Defines the theorem text style for each type of theorem to one of the three styles above
\newcounter{dummy}
\numberwithin{dummy}{section}
\theoremstyle{ocrenumbox}
\newtheorem{theoremeT}[dummy]{Theorem}
\newtheorem{problem}{Problem}[chapter]
\newtheorem{exerciseT}{Exercise}[chapter]
\theoremstyle{blacknumex}
\newtheorem{exampleT}{Example}[chapter]
\theoremstyle{blacknumbox}
\newtheorem{vocabulary}{Vocabulary}[chapter]
\newtheorem{definitionT}{Definition}[section]
\newtheorem{corollaryT}[dummy]{Corollary}
\theoremstyle{ocrenum}
\newtheorem{proposition}[dummy]{Proposition}

%----------------------------------------------------------------------------------------
%	DEFINITION OF COLORED BOXES
%----------------------------------------------------------------------------------------

\RequirePackage[framemethod=default]{mdframed} % Required for creating the theorem, definition, exercise and corollary boxes

% Theorem box
\newmdenv[skipabove=7pt,
skipbelow=7pt,
backgroundcolor=black!5,
linecolor=ocre,
innerleftmargin=5pt,
innerrightmargin=5pt,
innertopmargin=5pt,
leftmargin=0cm,
rightmargin=0cm,
innerbottommargin=5pt]{tBox}

% Exercise box
\newmdenv[skipabove=7pt,
skipbelow=7pt,
rightline=false,
leftline=true,
topline=false,
bottomline=false,
backgroundcolor=ocre!10,
linecolor=ocre,
innerleftmargin=5pt,
innerrightmargin=5pt,
innertopmargin=5pt,
innerbottommargin=5pt,
leftmargin=0cm,
rightmargin=0cm,
linewidth=4pt]{eBox}

% Definition box
\newmdenv[skipabove=7pt,
skipbelow=7pt,
rightline=false,
leftline=true,
topline=false,
bottomline=false,
linecolor=ocre,
innerleftmargin=5pt,
innerrightmargin=5pt,
innertopmargin=0pt,
leftmargin=0cm,
rightmargin=0cm,
linewidth=4pt,
innerbottommargin=0pt]{dBox}

% Corollary box
\newmdenv[skipabove=7pt,
skipbelow=7pt,
rightline=false,
leftline=true,
topline=false,
bottomline=false,
linecolor=gray,
backgroundcolor=black!5,
innerleftmargin=5pt,
innerrightmargin=5pt,
innertopmargin=5pt,
leftmargin=0cm,
rightmargin=0cm,
linewidth=4pt,
innerbottommargin=5pt]{cBox}

% Creates an environment for each type of theorem and assigns it a theorem text style from the "Theorem Styles" section above and a colored box from above
\newenvironment{theorem}{\begin{tBox}\begin{theoremeT}}{\end{theoremeT}\end{tBox}}
\newenvironment{exercise}{\begin{eBox}\begin{exerciseT}}{\hfill{\color{ocre}\tiny\ensuremath{\blacksquare}}\end{exerciseT}\end{eBox}}
\newenvironment{definition}{\begin{dBox}\begin{definitionT}}{\end{definitionT}\end{dBox}}
\newenvironment{example}{\begin{exampleT}}{\hfill{\tiny\ensuremath{\blacksquare}}\end{exampleT}}
\newenvironment{corollary}{\begin{cBox}\begin{corollaryT}}{\end{corollaryT}\end{cBox}}

%----------------------------------------------------------------------------------------
%	REMARK ENVIRONMENT
%----------------------------------------------------------------------------------------

\newenvironment{remark}{\par\vspace{10pt}\small % Vertical white space above the remark and smaller font size
\begin{list}{}{
\leftmargin=35pt % Indentation on the left
\rightmargin=25pt}\item\ignorespaces % Indentation on the right
\makebox[-2.5pt]{\begin{tikzpicture}[overlay]
\node[draw=ocre!60,line width=1pt,circle,fill=ocre!25,font=\sffamily\bfseries,inner sep=2pt,outer sep=0pt] at (-15pt,0pt){\textcolor{ocre}{R}};\end{tikzpicture}} % Orange R in a circle
\advance\baselineskip -1pt}{\end{list}\vskip5pt} % Tighter line spacing and white space after remark

%----------------------------------------------------------------------------------------
%	SECTION NUMBERING IN THE MARGIN
%----------------------------------------------------------------------------------------

\makeatletter
\renewcommand{\@seccntformat}[1]{\llap{\textcolor{ocre}{\csname the#1\endcsname}\hspace{1em}}}
\renewcommand{\section}{\@startsection{section}{1}{\z@}
{-4ex \@plus -1ex \@minus -.4ex}
{1ex \@plus.2ex }
{\normalfont\large\sffamily\bfseries}}
\renewcommand{\subsection}{\@startsection {subsection}{2}{\z@}
{-3ex \@plus -0.1ex \@minus -.4ex}
{0.5ex \@plus.2ex }
{\normalfont\sffamily\bfseries}}
\renewcommand{\subsubsection}{\@startsection {subsubsection}{3}{\z@}
{-2ex \@plus -0.1ex \@minus -.2ex}
{.2ex \@plus.2ex }
{\normalfont\small\sffamily\bfseries}}
\renewcommand\paragraph{\@startsection{paragraph}{4}{\z@}
{-2ex \@plus-.2ex \@minus .2ex}
{.1ex}
{\normalfont\small\sffamily\bfseries}}

%----------------------------------------------------------------------------------------
%	PART HEADINGS
%----------------------------------------------------------------------------------------

% numbered part in the table of contents
\newcommand{\@mypartnumtocformat}[2]{%
\setlength\fboxsep{0pt}%
\noindent\colorbox{ocre!20}{\strut\parbox[c][.7cm]{\ecart}{\color{ocre!70}\Large\sffamily\bfseries\centering#1}}\hskip\esp\colorbox{ocre!40}{\strut\parbox[c][.7cm]{\linewidth-\ecart-\esp}{\Large\sffamily\centering#2}}}%
%%%%%%%%%%%%%%%%%%%%%%%%%%%%%%%%%%
% unnumbered part in the table of contents
\newcommand{\@myparttocformat}[1]{%
\setlength\fboxsep{0pt}%
\noindent\colorbox{ocre!40}{\strut\parbox[c][.7cm]{\linewidth}{\Large\sffamily\centering#1}}}%
%%%%%%%%%%%%%%%%%%%%%%%%%%%%%%%%%%
\newlength\esp
\setlength\esp{4pt}
\newlength\ecart
\setlength\ecart{1.2cm-\esp}
\newcommand{\thepartimage}{}%
\newcommand{\partimage}[1]{\renewcommand{\thepartimage}{#1}}%
\def\@part[#1]#2{%
\ifnum \c@secnumdepth >-2\relax%
\refstepcounter{part}%
\addcontentsline{toc}{part}{\texorpdfstring{\protect\@mypartnumtocformat{\thepart}{#1}}{\partname~\thepart\ ---\ #1}}
\else%
\addcontentsline{toc}{part}{\texorpdfstring{\protect\@myparttocformat{#1}}{#1}}%
\fi%
\startcontents%
\markboth{}{}%
{\thispagestyle{empty}%
\begin{tikzpicture}[remember picture,overlay]%
\node at (current page.north west){\begin{tikzpicture}[remember picture,overlay]%
\fill[ocre!20](0cm,0cm) rectangle (\paperwidth,-\paperheight);
\node[anchor=north] at (4cm,-3.25cm){\color{ocre!40}\fontsize{220}{100}\sffamily\bfseries\@Roman\c@part};
\node[anchor=south east] at (\paperwidth-1cm,-\paperheight+1cm){\parbox[t][][t]{15cm}{
\printcontents{l}{0}{\setcounter{tocdepth}{1}}% El 15cm corresponde al ancho para el los contenidos de las partes del documento
}};
\node[anchor=north east] at (\paperwidth-1.5cm,-3.25cm){\parbox[t][][t]{15cm}{\strut\raggedleft\color{white}\fontsize{30}{30}\sffamily\bfseries#2}};
\end{tikzpicture}};
\end{tikzpicture}}%
\@endpart}
\def\@spart#1{%
\startcontents%
\phantomsection
{\thispagestyle{empty}%
\begin{tikzpicture}[remember picture,overlay]%
\node at (current page.north west){\begin{tikzpicture}[remember picture,overlay]%
\fill[ocre!20](0cm,0cm) rectangle (\paperwidth,-\paperheight);
\node[anchor=north east] at (\paperwidth-1.5cm,-3.25cm){\parbox[t][][t]{15cm}{\strut\raggedleft\color{white}\fontsize{30}{30}\sffamily\bfseries#1}};
\end{tikzpicture}};
\end{tikzpicture}}
\addcontentsline{toc}{part}{\texorpdfstring{%
\setlength\fboxsep{0pt}%
\noindent\protect\colorbox{ocre!40}{\strut\protect\parbox[c][.7cm]{\linewidth}{\Large\sffamily\protect\centering #1\quad\mbox{}}}}{#1}}%
\@endpart}
\def\@endpart{\vfil\newpage
\if@twoside
\if@openright
\null
\thispagestyle{empty}%
\newpage
\fi
\fi
\if@tempswa
\twocolumn
\fi}

%----------------------------------------------------------------------------------------
%	CHAPTER HEADINGS
%----------------------------------------------------------------------------------------

\newcommand{\thechapterimage}{}%
\newcommand{\chapterimage}[1]{\renewcommand{\thechapterimage}{#1}}%
\def\@makechapterhead#1{%
{\parindent \z@ \raggedright \normalfont
\ifnum \c@secnumdepth >\m@ne
\if@mainmatter
\begin{tikzpicture}[remember picture,overlay]
\node at (current page.north west)
{\begin{tikzpicture}[remember picture,overlay]
\node[anchor=north west,inner sep=0pt] at (0,0) {\includegraphics[width=\paperwidth]{\thechapterimage}};
\draw[anchor=west] (\Gm@lmargin,-6cm) node [line width=2pt,rounded corners=10pt,draw=ocre,fill=white,fill opacity=0.5,inner sep=15pt]{\strut\makebox[22cm]{}};
\draw[anchor=west] (\Gm@lmargin+.3cm,-6cm) node {\Large\sffamily\bfseries\color{black}\thechapter. #1\strut};
\end{tikzpicture}};
\end{tikzpicture}
\else
\begin{tikzpicture}[remember picture,overlay]
\node at (current page.north west)
{\begin{tikzpicture}[remember picture,overlay]
\node[anchor=north west,inner sep=0pt] at (0,0) {\includegraphics[width=\paperwidth]{\thechapterimage}};
\draw[anchor=west] (\Gm@lmargin,-6cm) node [line width=2pt,rounded corners=10pt,draw=ocre,fill=white,fill opacity=0.5,inner sep=15pt]{\strut\makebox[22cm]{}};
\draw[anchor=west] (\Gm@lmargin+.3cm,-6cm) node {\Large\sffamily\bfseries\color{black}#1\strut};
\end{tikzpicture}};
\end{tikzpicture}
\fi\fi\par\vspace*{170\p@}}}

%-------------------------------------------

% Encabezado de los capítulos

\def\@makeschapterhead#1{%
\begin{tikzpicture}[remember picture,overlay]
\node at (current page.north west)
{\begin{tikzpicture}[remember picture,overlay]
\node[anchor=north west,inner sep=0pt] at (0,0) {\includegraphics[width=\paperwidth]{\thechapterimage}};
\draw[anchor=west] (\Gm@lmargin,-6cm) node [line width=2pt,rounded corners=10pt,draw=ocre,fill=white,fill opacity=0.5,inner sep=15pt]{\strut\makebox[22cm]{}};
\draw[anchor=west] (\Gm@lmargin+.3cm,-6cm) node {\Large\sffamily\bfseries\color{black}#1\strut};
\end{tikzpicture}};
\end{tikzpicture}
\par\vspace*{170\p@}}
%\par\vspace*{270\p@}}
\makeatother

%----------------------------------------------------------------------------------------
%	HYPERLINKS IN THE DOCUMENTS
%----------------------------------------------------------------------------------------

\usepackage[pdftex]{hyperref}

%\definecolor{AzulEnlace}{RGB}{78,108,136}
\definecolor{AzulEnlace}{RGB}{0,145,209}
\definecolor{AzulEnlace}{RGB}{109,208,236}
\definecolor{AzulEnlace}{RGB}{31,112,191}

%31 112 191
%109 208 236
%0 152 212

\hypersetup{
	hidelinks,
	colorlinks=true,
	breaklinks=true,
	urlcolor= cyan,
	linkcolor=AzulEnlace,
	citecolor = AzulEnlace,
	bookmarksopen=false,
	pdftitle={Title},
	pdfauthor={Author}
}





%\hypersetup{hidelinks,backref=true,pagebackref=true,hyperindex=true,colorlinks=blue,breaklinks=true,urlcolor= ocre,bookmarks=true,bookmarksopen=false,pdftitle={Title},pdfauthor={Author}}

\usepackage{bookmark}
\bookmarksetup{
open,
numbered,
addtohook={%
\ifnum\bookmarkget{level}=0 % chapter
\bookmarksetup{bold}%
\fi
\ifnum\bookmarkget{level}=-1 % part
\bookmarksetup{color=ocre,bold}%
\fi
}
}


%% Para simulación de consola
\usepackage[most,many]{tcolorbox}

\tcbuselibrary{listings}

\usepackage{listings, color}
\definecolor{DarkGrey}{rgb}{0.1,0.1,0.1}
\definecolor{AzulWin}{RGB}{22,43,58}

\newtcblisting{commandshell}{
colback=black,
colupper=white,
colframe=white!55!black,
listing only,
listing options={language=sh},
every listing line={\textcolor{white}{\small\ttfamily\bfseries \$ }}}

\newtcblisting{commandshellroot}{
colback=black,
colupper=yellow,
colframe=white!75!black,
listing only,
listing options={language=sh},
every listing line={\textcolor{white}{\small\ttfamily\bfseries \# }}
}

\newtcblisting{powerShell}{
	colback=AzulWin,
	colupper=white,
	colframe=white!55!black,
	listing only,}

\newtcblisting{textshell}{
	colback=lightgray,
	colupper=black,
	colframe=white!75!black,
	listing only,
	listing options={language=sh},
	every listing line={\textcolor{white}},
	listing options={
		basicstyle=\small\ttfamily,
		breaklines=true,
		columns=fullflexible
	}
}

\lstdefinestyle{cmdWin} {
    language=VBScript,
    backgroundcolor=\color{DarkGrey},
    keywordstyle=\color{BlueViolet}\bfseries,
    commentstyle=\color{Grey},
    stringstyle=\color{Red},
    showstringspaces=false,
    basicstyle=\small\color{white},
    numbers=none,
    captionpos=b,
    tabsize=4,
    breaklines=true
}

%\lstdefinestyle{cmdWin} {
%	language=Bash,
%	backgroundcolor=\color{DarkGrey},
%	keywordstyle=\color{BlueViolet}\bfseries,
%	commentstyle=\color{Grey},
%	stringstyle=\color{Red},
%	showstringspaces=false,
%	basicstyle=\small\color{white},
%	numbers=none,
%	captionpos=b,
%	tabsize=4,
%	breaklines=true
%}

\lstdefinelanguage{json}{
	basicstyle=\normalfont\ttfamily,
	basicstyle=\scriptsize\ttfamily,
	numbers=left,
	numberstyle=\scriptsize,
	stepnumber=1,
	numbersep=7pt,
	showstringspaces=false,
	breaklines=true,
	frame=lines,
	extendedchars=true,
	inputencoding=utf8,
	backgroundcolor=\color{background},
	literate=
	*{0}{{{\color{numb}0}}}{1}
	{1}{{{\color{numb}1}}}{1}
	{2}{{{\color{numb}2}}}{1}
	{3}{{{\color{numb}3}}}{1}
	{4}{{{\color{numb}4}}}{1}
	{5}{{{\color{numb}5}}}{1}
	{6}{{{\color{numb}6}}}{1}
	{7}{{{\color{numb}7}}}{1}
	{8}{{{\color{numb}8}}}{1}
	{9}{{{\color{numb}9}}}{1}
	{:}{{{\color{punct}{:}}}}{1}
	{,}{{{\color{punct}{,}}}}{1}
	{\{}{{{\color{delim}{\{}}}}{1}
	{\}}{{{\color{delim}{\}}}}}{1}
	{[}{{{\color{delim}{[}}}}{1}
	{]}{{{\color{delim}{]}}}}{1}
	{á}{{\'a}}1
	{é}{{\'e}}1
	{í}{{\'i}}1
	{ó}{{\'o}}1
	{ú}{{\'u}}1
	{ñ}{{\~n}}1,
}

\usepackage{listings}

\colorlet{punct}{red!60!black}
\definecolor{background}{HTML}{EEEEEE}
\definecolor{delim}{RGB}{20,105,176}
\colorlet{numb}{magenta!60!black}






%%% By Luixip
%\usepackage{graphicx}
\usepackage{fancybox}
\usepackage{color}
%\usepackage{float} % para usar [H] en figuras y tablas
\usepackage[none]{hyphenat}
%\usepackage{amsmath}
%\usepackage{amssymb}
%\usepackage{wrapfig}
\usepackage{wrapfig, blindtext}
\usepackage[section]{placeins}
\usepackage[export]{adjustbox}
\usepackage{pdfpages}
%\usepackage{acronym}

\usepackage{ragged2e} % para justicar texto

%\usepackage[table]{xcolor}% http://ctan.org/pkg/xcolor
%\usepackage[table]{xcolor}

%\usepackage{colortbl}


%\usepackage{tabularx}
%\usepackage{multirow, array} % para las tablas
%\usepackage{array} % para las tablas
%\usepackage{longtable}
%\usepackage[footnotesize,labelfont=bf,up,textfont=it,up]{caption} % Custom captions under/above floats in tables or figures
\usepackage[footnotesize,labelfont=bf,up,textfont=it,up, center]{caption} % Custom captions under/above floats in tables or figures

%\usepackage[hang, small,labelfont=bf,up,textfont=it,up]{caption} % Custom captions under/above floats in tables or figures
%\usepackage{booktabs} % Horizontal rules in tables
\usepackage{float} % Required for tables and figures in the multi-column environment - they
%\usepackage{graphicx} % paquete que permite introducir imágenes
%\usepackage{booktabs} % Horizontal rules in tables
%\usepackage{float} % Required for tables and figures in the multi-column environment - they

\usepackage{ftnxtra} % Permite color los pie de página en el Caption de las figuras.

%\usepackage{sectsty}

% Number equations within sections (i.e. 1.1, 1.2, 2.1, 2.2 instead of 1, 2, 3, 4)
\numberwithin{equation}{section}
% Number figures within sections (i.e. 1.1, 1.2, 2.1, 2.2 instead of 1, 2, 3, 4)
\numberwithin{figure}{section}
% Number tables within sections (i.e. 1.1, 1.2, 2.1, 2.2 instead of 1, 2, 3, 4)
\numberwithin{table}{section}

\usepackage{parskip}
%%%%%%%%%%%\setlength\parindent{0pt} % Removes all indentation from paragraphs - comment this line for an assignment with lots of text

\definecolor{AzulTitulo}{RGB}{18,42,68}



%% Manejo de acrónimos
\usepackage[acronym]{glossaries}
\makeglossaries
\makeglossaries
\glossarystyle{altlistgroup}
%% Definción de términos (Glosario)

\newglossaryentry{ciberinfraestructura}
{
	name=Ciberinfraestructura,
	description=
	{
		Concepto conocido en inglés como \textit{Cyberinfrastructure}, el cual según \textcite{Stewart2019} continúa ampliándose y hace referencia a "la integración de supercomputadoras, recursos de datos, visualización y personas que extienden el impacto y la utilidad de las tecnologías de la información"
	}
}


\newglossaryentry{EcosistemaTecnoloogicoDeApoyoALaInvestigacioon}
{
	name={Ecosistema Tecnológico de Apoyo a la Investigación},
	description=
	{
		Este concepto, es acuñado en el presente trabajo y hace referencia a la comunidad investigativa de una institución, considerando sus relaciones y procesos apalancados en el uso de \acrfull{RSI}. Este concepto es extrapolado desde el ecosistema biológico y es similar al concepto de ecosistema tecnológico para la educación indicado por \parencite{Garcia-Holgado2018}
	}
}

\newglossaryentry{computacionEnLaNube}
{
	name=Computación en la nube,
	description=
	{
		Concepto referido en inglés como Cloud Computing y según el \acrfull{NIST} se refiere a “un modelo para permitir el acceso ubicuo, conveniente y por demanda de un conjunto compartido de recursos informáticos configurables (redes, servidores, almacenamiento, aplicaciones y servicios) que se pueden aprovisionar y liberar rápidamente con un mínimo de esfuerzo de gestión o de interacción con el proveedor de servicios” \parencite{Mell2011}
	}
}

\newglossaryentry{Interoperabilidad}
{
	name=Interoperabilidad,
	description=
	{
		Este concepto según \textcite{Lara2007} posee diversas acepciones según el ámbito de aplicación y en el ámbito de la informática se define como "la capacidad del software y del hardware perteneciente a diferentes máquinas y de diferentes marcas comerciales para compartir datos". En ese mismo sentido la Comisión Europea citada en el Libro Blanco de la interoperabilidad de gobierno electrónico para América Latina y el Caribe \parencite{CEPAL2007} la definen como “la habilidad de los sistemas TIC, y de los procesos de negocio que ellas soportan, de intercambiar datos, posibilitar información y conocimiento”; también se indica que la interoperabilidad se desarrolla teniendo presente los aspectos semánticos, organizacionales, técnicos y de gobernanza
	}
}

\newglossaryentry{modelo}
{
	name=modelo,
	description=
	{
		Este término posee diversas acepciones dependiendo del ámbito de aplicación, sin embargo, para el contexto del presente trabajo y según lo indica el diccionario de la lengua española \parencite{RAE_Modelo2020}, el término modelo se define como "Arquetipo o punto de referencia para imitarlo o reproducirlo"
	}
}

\newglossaryentry{PortalCientifico}
{
	name={portal científico},
	description={
		Ver \gls{ScienceGateway}.
	}
}

\newglossaryentry{def:RSI}
{
	name={Recursos y Servicios Informaticos},
	description={Conjunto heterogéneo que comprende múltiples elementos entre los que destacan: datos (por ejemplo, el resultado de experimentos), sistemas especializados (software fuertemente acoplado a procesos técnicos), equipos de laboratorio, bases de datos, librerías, sistemas de computación distribuidos y sistemas de computación en la nube, entre otros
	}	
}


\newglossaryentry{ScienceGateway}
{
	name={Science Gateway},
	description={
		El concepto \acrfull{SG} posee amplia aceptación en países como Estados Unidos y Canadá y es definido según los trabajos de \textcite{Barker2019} y \textcite{Wilkins-Diehr2013} como "\textit{sistemas de información empresarial basados en la web que proporcionan a los científicos acceso fácil y personalizado a las colecciones de datos específicos de la comunidad, herramientas computacionales y servicios de colaboración sobre infraestructuras electrónicas}". El concepto \acrshort{SG} también abarca a otros conceptos tales como "\textit{laboratorios virtuales}" y los "\textit{ambientes virtuales de investigación}", en inglés Virtual Research Enviroment
	}
}


\newglossaryentry{VirtualResearchEnvironment}
{
	name={virtual research environment},
	description={
		Este concepto conocido como \acrfull{VRE} según \textcite{Barker2019} tiene mayor aceptación en Australia y es considerado como una abreviatura de las herramientas y tecnologías que necesita el investigador para realizar su investigación, interactuar con otros investigadores y para hacer uso de los recursos y las infraestructuras técnicas disponibles tanto a nivel local como nacional
	}
}

\newglossaryentry{Virtualizacioon}
{
	name={virtualización},
	description={
		Este término es considerado como la combinación de diferentes tecnologías ofreciendo ventajas como el apoyo para la gestión de grandes volúmenes de datos, la posibilidad de escalar rápidamente y el aprovechamiento de grandes capacidades de cómputo; técnicamente y según se señala en los estudios de \textcite{Kusnetzky2011}  y \textcite{AbdElRahem2016}, la virtualización permite abstraer aplicaciones y los componentes subyacentes de hardware para ser presentados como una vista lógica o virtual de estos recursos. Esta vista lógica puede ser en ocasiones notablemente diferente a la vista física, la cual por lo general se construye a partir del exceso de recursos de computación, tales como el poder de procesamiento, la memoria, la capacidad de almacenamiento o incluso el ancho de banda \parencite{Stallings2015}
	}
}




% abcdefghijklmnñopqrstuvwxyz

%% Definición de acrónimos

\newacronym{ALyC}
	{\textit{ALyC}}
	{\textit{América Latina y el Caribe}}

\newacronym{CDETI}
    {\textit{CDETI}}
    {\textit{Comité para el Direccionamiento Estratégico de TI}}

\newacronym{CMMI}
    {\textit{CMMI}}
    {\textit{Capability Maturity Model Integration}}

\newacronym{CPD}
	{\textit{CPD}}
	{\textit{Centro de Procesamiento de Datos}}

\newacronym{ETAI}
    {\textit{ETAI}}
    {\textit{Ecosistema Tecnológico de Apoyo a la Investigación}}

%\newacronym{ETAIs}
%{\textit{ETAIs}}
%{\textit{Ecosistemas Tecnológicos de Apoyo a la Investigación}}

\newacronym{GSTI}
    {\textit{GSTI}}
    {\textit{Gestión de Servicios de Tecnologías de la Información}}

\newacronym{IES}
    {\textit{IES}}
    {\textit{Instituciones de Educación Superior}}

\newacronym{ISO}
    {\textit{\textit{ISO}}}
    {\textit{International Organization for Standarization}}

\newacronym{IT}
	{\textit{IT}}
	{\textit{Information Technology}}

\newacronym{ITIL}
    {\textit{ITIL}}
    {\textit{Information Technology Infrastructure Library}}

\newacronym{ITSM}
    {\textit{ITSM}}
    {\textit{Information Technology Service Management}}

%\newacronym{ISO/IEC 20000}{ISO/IEC 20000}{ISO/IEC 20000}

\newacronym{GRSI-ETAI}
    {\textit{GRSI-ETAI}}
    {\textit{Modelo de Referencia para la Gestión de Recursos y Servicios Informáticos en Ecosistemas Tecnológicos de Apoyo a la Investigación}
    }


\newacronym{NIST}
    {\textit{NIST}}
    {\textit{National Institute of Standards and Technology}}

\newacronym{NREN}
    {\textit{NREN}}
    {National Research and Education Network}

\newacronym{RENATA}
    {\textit{RENATA}}
    {Corporación Red Nacional Académica de Tecnología Avanzada}

\newacronym{RSI}
    {\textit{RSI}}
    {\gls{def:RSI}}
%    {\textit{Recursos y Servicios Informáticos}}

\newacronym{RI}
    {\textit{RI}}
    {\textit{Recurso Informático}}

\newacronym{SG}
    {\textit{SG}}
    {\textit{Science Gateway}}

\newacronym{SGCI}
    {\textit{SGCI}}
    {\textit{Science Gateway Community Intitute}}

\newacronym{SI}
    {\textit{SI}}
    {\textit{Servicio Informático}}

\newacronym{SMS}
    {\textit{SMS}}
    {\textit{Systematic Mapping Study}}

\newacronym{SPS}
    {\textit{SPS}}
    {\textit{Selected Primary Study}}

\newacronym{SUE}
    {\textit{SUE}}
    {\textit{Sistema Universitario Estatal}}

\newacronym{SVC}
    {\textit{SVC}}
    {\textit{Service Value Chain}}

\newacronym{SVS}
    {\textit{SVS}}
    {\textit{Service Value Syste}m}

\newacronym{TI}
    {\textit{TI}}
    {\textit{Tecnologías de la Información}}

\newacronym{TERS}
    {\textit{TERS}}
    {\textit{Technological Ecosystem for Research Support}}

\newacronym{VRE}
    {\textit{VRE}}
    {\textit{\gls{VirtualResearchEnvironment}}}

\newacronym{VL}
    {\textit{VL}}
    {\textit{Virtual Laboratories}}



% Acónimos de los dominios COBIT 2019
\newacronym{COBIT}
    {\textit{COBIT}}
    {\textit{Control Objetives for Information and related Technology}}

\newacronym{GETI}
    {\textit{GETI}}
    {\textit{Gobierno Empresarial de Tecnologías de la Información}}

\newacronym{EDM}
    {\textit{EDM}}
    {\textit{Evaluate, Direct and Monitor}} % Dominio Cobit

\newacronym{APO}
    {\textit{APO}}
    {\textit{Align, Plan and Organize}} % Dominio Cobit

\newacronym{BAI}
    {\textit{BAI}}
    {\textit{Build, Acquire and Implement}} % Dominio Cobit

\newacronym{DSS}
    {\textit{DSS}}
    {\textit{Deliver, Service and Support}} % Dominio Cobit

\newacronym{MEA}
    {\textit{MEA}}
    {\textit{Monitor, Evaluate and Assess}} % Dominio Cobit

\newacronym{GMO}
    {\textit{GMO}}
    {Governance and Management Objectives} %

\newacronym{OGG}
    {\textit{OGG}}
    {\textit{Objetivos para el Gobierno y la Gestión}} %


\renewcommand{\glossaryname}{Glosario}
\renewcommand{\acronymname}{Acrónimos}


%%\usepackage{hyperref} % By luixip


% para Cambiar "y col." por "et al."
\DefineBibliographyStrings{spanish}{
    andothers = {et\addabbrvspace al\adddot},
    andmore   = {et\addabbrvspace al\adddot}
}

% Para controlar la produndida de las secciones mostradas en la tabla de contenido
\setcounter{tocdepth}{2}
% Para enumerar subsecciones (nivel de profundidad)
\setcounter{secnumdepth}{3}


% Indicando la distancia entre párrafos (ojo!!)
%\setlength{\parskip}{8mm}
\usepackage{titlesec}

% Definition of \subparagraph starting new line after heading
\titleformat{\subparagraph}
{\normalfont\normalsize\bfseries}{\thesubparagraph}{1em}{}
\titlespacing*{\subparagraph}{\parindent}{1.25ex plus 1ex minus .2ex}{.75ex plus .1ex}

% Create Roman Numeral Labelled Annexes
\newcommand{\annexname}{Anexos}
\makeatletter % treat @ as a letter instead of a control word.
\newcommand\annex{\par
    \setcounter{chapter}{0}
    \setcounter{section}{0}
    \renewcommand\appendixname{Anexos}
    \renewcommand\appendixpagename{Anexos}
    \renewcommand{\appendixtocname}{Anexos}
    \gdef\@chapapp{\annexname}
%    \gdef\thechapter{\@Roman\c@chapter}
    \gdef\thechapter{\@Alph\c@chapter}
    \renewcommand{\theHchapter}{\annexname.\thechapter}
    \addappheadtotoc
}
\makeatother

\usepackage{alertmessage}

\hyphenation{me-to-do-ló-gi-cos im-ple-men-ta-dos pre-li-mi-nar-men-te con-si-de-ra-cio-nes}


% Luixip: Evitar partir las palabras en latex
%--------------------------------------------
\pretolerance=10000
\tolerance=10000
%---------------------------------------------