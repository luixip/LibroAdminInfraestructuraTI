\documentclass[12pt,fleqn,openany,letter]{book}
\usepackage{appendix}
\usepackage{minted}
\input{estructura.tex}


%%% By Luixip
%\usepackage{graphicx}
\usepackage{fancybox}
\usepackage{color}
%\usepackage{float} % para usar [H] en figuras y tablas
\usepackage[none]{hyphenat}
%\usepackage{amsmath}
%\usepackage{amssymb}
%\usepackage{wrapfig}
\usepackage{wrapfig, blindtext}
\usepackage[section]{placeins}
\usepackage[export]{adjustbox}
\usepackage{pdfpages}
%\usepackage{acronym}

\usepackage{ragged2e} % para justicar texto

%\usepackage[table]{xcolor}% http://ctan.org/pkg/xcolor
%\usepackage[table]{xcolor}

%\usepackage{colortbl}


%\usepackage{tabularx}
%\usepackage{multirow, array} % para las tablas
%\usepackage{array} % para las tablas
%\usepackage{longtable}
%\usepackage[footnotesize,labelfont=bf,up,textfont=it,up]{caption} % Custom captions under/above floats in tables or figures
\usepackage[footnotesize,labelfont=bf,up,textfont=it,up, center]{caption} % Custom captions under/above floats in tables or figures

%\usepackage[hang, small,labelfont=bf,up,textfont=it,up]{caption} % Custom captions under/above floats in tables or figures
%\usepackage{booktabs} % Horizontal rules in tables
\usepackage{float} % Required for tables and figures in the multi-column environment - they
%\usepackage{graphicx} % paquete que permite introducir imágenes
%\usepackage{booktabs} % Horizontal rules in tables
%\usepackage{float} % Required for tables and figures in the multi-column environment - they

\usepackage{ftnxtra} % Permite color los pie de página en el Caption de las figuras.

%\usepackage{sectsty}

% Number equations within sections (i.e. 1.1, 1.2, 2.1, 2.2 instead of 1, 2, 3, 4)
\numberwithin{equation}{section}
% Number figures within sections (i.e. 1.1, 1.2, 2.1, 2.2 instead of 1, 2, 3, 4)
\numberwithin{figure}{section}
% Number tables within sections (i.e. 1.1, 1.2, 2.1, 2.2 instead of 1, 2, 3, 4)
\numberwithin{table}{section}

\usepackage{parskip}
%%%%%%%%%%%\setlength\parindent{0pt} % Removes all indentation from paragraphs - comment this line for an assignment with lots of text

\definecolor{AzulTitulo}{RGB}{18,42,68}



%% Manejo de acrónimos
\usepackage[acronym]{glossaries}
\makeglossaries
\makeglossaries
\glossarystyle{altlistgroup}
%% Definción de términos (Glosario)

\newglossaryentry{ciberinfraestructura}
{
	name=Ciberinfraestructura,
	description=
	{
		Concepto conocido en inglés como \textit{Cyberinfrastructure}, el cual según \textcite{Stewart2019} continúa ampliándose y hace referencia a "la integración de supercomputadoras, recursos de datos, visualización y personas que extienden el impacto y la utilidad de las tecnologías de la información"
	}
}


\newglossaryentry{EcosistemaTecnoloogicoDeApoyoALaInvestigacioon}
{
	name={Ecosistema Tecnológico de Apoyo a la Investigación},
	description=
	{
		Este concepto, es acuñado en el presente trabajo y hace referencia a la comunidad investigativa de una institución, considerando sus relaciones y procesos apalancados en el uso de \acrfull{RSI}. Este concepto es extrapolado desde el ecosistema biológico y es similar al concepto de ecosistema tecnológico para la educación indicado por \parencite{Garcia-Holgado2018}
	}
}

\newglossaryentry{computacionEnLaNube}
{
	name=Computación en la nube,
	description=
	{
		Concepto referido en inglés como Cloud Computing y según el \acrfull{NIST} se refiere a “un modelo para permitir el acceso ubicuo, conveniente y por demanda de un conjunto compartido de recursos informáticos configurables (redes, servidores, almacenamiento, aplicaciones y servicios) que se pueden aprovisionar y liberar rápidamente con un mínimo de esfuerzo de gestión o de interacción con el proveedor de servicios” \parencite{Mell2011}
	}
}

\newglossaryentry{Interoperabilidad}
{
	name=Interoperabilidad,
	description=
	{
		Este concepto según \textcite{Lara2007} posee diversas acepciones según el ámbito de aplicación y en el ámbito de la informática se define como "la capacidad del software y del hardware perteneciente a diferentes máquinas y de diferentes marcas comerciales para compartir datos". En ese mismo sentido la Comisión Europea citada en el Libro Blanco de la interoperabilidad de gobierno electrónico para América Latina y el Caribe \parencite{CEPAL2007} la definen como “la habilidad de los sistemas TIC, y de los procesos de negocio que ellas soportan, de intercambiar datos, posibilitar información y conocimiento”; también se indica que la interoperabilidad se desarrolla teniendo presente los aspectos semánticos, organizacionales, técnicos y de gobernanza
	}
}

\newglossaryentry{modelo}
{
	name=modelo,
	description=
	{
		Este término posee diversas acepciones dependiendo del ámbito de aplicación, sin embargo, para el contexto del presente trabajo y según lo indica el diccionario de la lengua española \parencite{RAE_Modelo2020}, el término modelo se define como "Arquetipo o punto de referencia para imitarlo o reproducirlo"
	}
}

\newglossaryentry{PortalCientifico}
{
	name={portal científico},
	description={
		Ver \gls{ScienceGateway}.
	}
}

\newglossaryentry{def:RSI}
{
	name={Recursos y Servicios Informaticos},
	description={Conjunto heterogéneo que comprende múltiples elementos entre los que destacan: datos (por ejemplo, el resultado de experimentos), sistemas especializados (software fuertemente acoplado a procesos técnicos), equipos de laboratorio, bases de datos, librerías, sistemas de computación distribuidos y sistemas de computación en la nube, entre otros
	}	
}


\newglossaryentry{ScienceGateway}
{
	name={Science Gateway},
	description={
		El concepto \acrfull{SG} posee amplia aceptación en países como Estados Unidos y Canadá y es definido según los trabajos de \textcite{Barker2019} y \textcite{Wilkins-Diehr2013} como "\textit{sistemas de información empresarial basados en la web que proporcionan a los científicos acceso fácil y personalizado a las colecciones de datos específicos de la comunidad, herramientas computacionales y servicios de colaboración sobre infraestructuras electrónicas}". El concepto \acrshort{SG} también abarca a otros conceptos tales como "\textit{laboratorios virtuales}" y los "\textit{ambientes virtuales de investigación}", en inglés Virtual Research Enviroment
	}
}


\newglossaryentry{VirtualResearchEnvironment}
{
	name={virtual research environment},
	description={
		Este concepto conocido como \acrfull{VRE} según \textcite{Barker2019} tiene mayor aceptación en Australia y es considerado como una abreviatura de las herramientas y tecnologías que necesita el investigador para realizar su investigación, interactuar con otros investigadores y para hacer uso de los recursos y las infraestructuras técnicas disponibles tanto a nivel local como nacional
	}
}

\newglossaryentry{Virtualizacioon}
{
	name={virtualización},
	description={
		Este término es considerado como la combinación de diferentes tecnologías ofreciendo ventajas como el apoyo para la gestión de grandes volúmenes de datos, la posibilidad de escalar rápidamente y el aprovechamiento de grandes capacidades de cómputo; técnicamente y según se señala en los estudios de \textcite{Kusnetzky2011}  y \textcite{AbdElRahem2016}, la virtualización permite abstraer aplicaciones y los componentes subyacentes de hardware para ser presentados como una vista lógica o virtual de estos recursos. Esta vista lógica puede ser en ocasiones notablemente diferente a la vista física, la cual por lo general se construye a partir del exceso de recursos de computación, tales como el poder de procesamiento, la memoria, la capacidad de almacenamiento o incluso el ancho de banda \parencite{Stallings2015}
	}
}




% abcdefghijklmnñopqrstuvwxyz

%% Definición de acrónimos

\newacronym{ALyC}
	{\textit{ALyC}}
	{\textit{América Latina y el Caribe}}

\newacronym{CDETI}
    {\textit{CDETI}}
    {\textit{Comité para el Direccionamiento Estratégico de TI}}

\newacronym{CMMI}
    {\textit{CMMI}}
    {\textit{Capability Maturity Model Integration}}

\newacronym{CPD}
	{\textit{CPD}}
	{\textit{Centro de Procesamiento de Datos}}

\newacronym{ETAI}
    {\textit{ETAI}}
    {\textit{Ecosistema Tecnológico de Apoyo a la Investigación}}

%\newacronym{ETAIs}
%{\textit{ETAIs}}
%{\textit{Ecosistemas Tecnológicos de Apoyo a la Investigación}}

\newacronym{GSTI}
    {\textit{GSTI}}
    {\textit{Gestión de Servicios de Tecnologías de la Información}}

\newacronym{IES}
    {\textit{IES}}
    {\textit{Instituciones de Educación Superior}}

\newacronym{ISO}
    {\textit{\textit{ISO}}}
    {\textit{International Organization for Standarization}}

\newacronym{IT}
	{\textit{IT}}
	{\textit{Information Technology}}

\newacronym{ITIL}
    {\textit{ITIL}}
    {\textit{Information Technology Infrastructure Library}}

\newacronym{ITSM}
    {\textit{ITSM}}
    {\textit{Information Technology Service Management}}

%\newacronym{ISO/IEC 20000}{ISO/IEC 20000}{ISO/IEC 20000}

\newacronym{GRSI-ETAI}
    {\textit{GRSI-ETAI}}
    {\textit{Modelo de Referencia para la Gestión de Recursos y Servicios Informáticos en Ecosistemas Tecnológicos de Apoyo a la Investigación}
    }


\newacronym{NIST}
    {\textit{NIST}}
    {\textit{National Institute of Standards and Technology}}

\newacronym{NREN}
    {\textit{NREN}}
    {National Research and Education Network}

\newacronym{RENATA}
    {\textit{RENATA}}
    {Corporación Red Nacional Académica de Tecnología Avanzada}

\newacronym{RSI}
    {\textit{RSI}}
    {\gls{def:RSI}}
%    {\textit{Recursos y Servicios Informáticos}}

\newacronym{RI}
    {\textit{RI}}
    {\textit{Recurso Informático}}

\newacronym{SG}
    {\textit{SG}}
    {\textit{Science Gateway}}

\newacronym{SGCI}
    {\textit{SGCI}}
    {\textit{Science Gateway Community Intitute}}

\newacronym{SI}
    {\textit{SI}}
    {\textit{Servicio Informático}}

\newacronym{SMS}
    {\textit{SMS}}
    {\textit{Systematic Mapping Study}}

\newacronym{SPS}
    {\textit{SPS}}
    {\textit{Selected Primary Study}}

\newacronym{SUE}
    {\textit{SUE}}
    {\textit{Sistema Universitario Estatal}}

\newacronym{SVC}
    {\textit{SVC}}
    {\textit{Service Value Chain}}

\newacronym{SVS}
    {\textit{SVS}}
    {\textit{Service Value Syste}m}

\newacronym{TI}
    {\textit{TI}}
    {\textit{Tecnología Informática}}

\newacronym{TERS}
    {\textit{TERS}}
    {\textit{Technological Ecosystem for Research Support}}

\newacronym{VRE}
    {\textit{VRE}}
    {\textit{\gls{VirtualResearchEnvironment}}}

\newacronym{VL}
    {\textit{VL}}
    {\textit{Virtual Laboratories}}



% Acónimos de los dominios COBIT 2019
\newacronym{COBIT}
    {\textit{COBIT}}
    {\textit{Control Objetives for Information and related Technology}}

\newacronym{GETI}
    {\textit{GETI}}
    {\textit{Gobierno Empresarial de Tecnologías de la Información}}

\newacronym{EDM}
    {\textit{EDM}}
    {\textit{Evaluate, Direct and Monitor}} % Dominio Cobit

\newacronym{APO}
    {\textit{APO}}
    {\textit{Align, Plan and Organize}} % Dominio Cobit

\newacronym{BAI}
    {\textit{BAI}}
    {\textit{Build, Acquire and Implement}} % Dominio Cobit

\newacronym{DSS}
    {\textit{DSS}}
    {\textit{Deliver, Service and Support}} % Dominio Cobit

\newacronym{MEA}
    {\textit{MEA}}
    {\textit{Monitor, Evaluate and Assess}} % Dominio Cobit

\newacronym{GMO}
    {\textit{GMO}}
    {Governance and Management Objectives} %

\newacronym{OGG}
    {\textit{OGG}}
    {\textit{Objetivos para el Gobierno y la Gestión}} %


\renewcommand{\glossaryname}{Glosario}
\renewcommand{\acronymname}{Acrónimos}


%%\usepackage{hyperref} % By luixip


% para Cambiar "y col." por "et al."
\DefineBibliographyStrings{spanish}{
    andothers = {et\addabbrvspace al\adddot},
    andmore   = {et\addabbrvspace al\adddot}
}

% Para controlar la produndida de las secciones mostradas en la tabla de contenido
\setcounter{tocdepth}{2}
% Para enumerar subsecciones (nivel de profundidad)
\setcounter{secnumdepth}{3}


% Indicando la distancia entre párrafos (ojo!!)
%\setlength{\parskip}{8mm}
\usepackage{titlesec}

% Definition of \subparagraph starting new line after heading
\titleformat{\subparagraph}
{\normalfont\normalsize\bfseries}{\thesubparagraph}{1em}{}
\titlespacing*{\subparagraph}{\parindent}{1.25ex plus 1ex minus .2ex}{.75ex plus .1ex}

% Create Roman Numeral Labelled Annexes
\newcommand{\annexname}{Anexos}
\makeatletter % treat @ as a letter instead of a control word.
\newcommand\annex{\par
    \setcounter{chapter}{0}
    \setcounter{section}{0}
    \renewcommand\appendixname{Anexos}
    \renewcommand\appendixpagename{Anexos}
    \renewcommand{\appendixtocname}{Anexos}
    \gdef\@chapapp{\annexname}
%    \gdef\thechapter{\@Roman\c@chapter}
    \gdef\thechapter{\@Alph\c@chapter}
    \renewcommand{\theHchapter}{\annexname.\thechapter}
    \addappheadtotoc
}
\makeatother

\usepackage{alertmessage}

\hyphenation{me-to-do-ló-gi-cos im-ple-men-ta-dos pre-li-mi-nar-men-te con-si-de-ra-cio-nes}


% Luixip: Evitar partir las palabras en latex
%--------------------------------------------
\pretolerance=10000
\tolerance=10000
%---------------------------------------------