\newpage
%~\vfill
\chapterimage{fondoSecciones} % Table of contents heading image
\thispagestyle{fancy}
{\huge \textbf{Resumen}}\\

%\alertwarning{Revisar información comentada}


Este libro presenta una conceptualización acerca de la infraestructura de Tecnología Informática (\acrshort{TI}) y su aporte en la entrega de valor para la gestión organizacional actual. De esta forma, se brinda información para fortalecer las capacidades laborales a los profesionales de TI.

En particular el contenido está enmarcado en el área curricular \textit{Infraestructura de Tecnología Informática} del programa académico \textit{Ingeniería de Sistemas y Computación}, adscrito a la \textit{Facultad de Ingeniería} de la \textit{Universidad del Quindío} en Armenia, Colombia.

El contenido presenta aspectos conceptuales que le permiten al lector comprender los principios y funcionalidades de la administración de la infraestructura de TI, enfocados en estándares y buenas prácticas que buscan mantener la operación de los componentes de TI según los requerimientos de las organizaciones y por esto, se incluyen temas como la identificación de necesidades organizacionales; el diseño y planeación de soluciones; la implantación y gestión de la infraestructura de TI; al igual que su monitorización.

Incluye aspectos conceptuales que le permiten al estudiante comprender los
principios y funcionalidades de la Administración de la Infraestructura Informática (TI), basados en referentes de industria, estándares y buenas prácticas que buscan mantener la operación de los componentes de TI según las necesidades de las
organizaciones.

En este sentido, aspectos como la identificación de necesidades
organizacionales de TI; el diseño y planeación de soluciones; la implantación y operación;
al igual que la monitorización de la Infraestructura de TI, hacen parte del conocimiento
esencial abordado y que complmenta las compentencias de los profesioanles de TI de un Ingeniero de Sistemas y Computación.


%Los aspectos centrales del libro consideran conceptos relacionados con las redes de informáticas, para lo cual se toma como referente al modelo TCP/IP; además, también se incluyen los conceptos relacionados con los sistemas de operativos que representan diversas elemetnos del modelo Cliente/Servidor. Todos estos conceptos se ilustran en ejemplos que tiene implemetaciones en prototipos funcionales implementados a través de tecnologías de virtualización como los hipervisores (tipo 1 y 2) y  contenedores.






