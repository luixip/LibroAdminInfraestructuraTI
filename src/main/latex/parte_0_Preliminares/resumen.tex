\newpage
%~\vfill
\chapterimage{fondoSecciones} % Table of contents heading image
\thispagestyle{fancy}
{\huge \textbf{Resumen}}\\

%\alertwarning{Revisar información comentada}

%El trabajo presentado, se centra en las tecnologías de virtualización y busca  realizar una revisión bibliográfica que permita la descripción de fundamentos, contexto histórico y ámbitos de aplicación de estas tecnologías. Todo esto para la construcción de un conjunto de recursos educativos orientados a la enseñanza de las tecnologías de virtualización, buscando un aporte significativo a la docencia, especialmente en los espacios académicos del área de infraestructura de \gls{TI} del programa de Ingeniería de Sistemas y Computación de la Facultad de Ingeniería de la Universidad del Quindío. \gls{ETAI} , \gls{SG}. \\



%La virtualización según lo señala \textcite{Goldworm2007} y \textcite{Kusnetzky2011} es una mezcla de tecnologías que permiten realizar una abstracción del hardware presentándolo como una vista lógica del mismo, lo que permite que múltiples “máquinas virtuales” se ejecuten en una misma máquina física (o real). Esta situación trae consigo un esquema de trabajo con mayor flexibilidad, que a su vez facilita la adaptación de la infraestructura de TI hacia las necesidades de las organizaciones. El concepto de tecnologías de virtualización utilizado en este trabajo, está suscrito en la taxonomía descrita por \textcite{Pessolani2012}, que a su vez cumple los requisitos asociados a la virtualización concebidos por \textcite{Popek1974}. Partiendo de dicha taxonomía, esta propuesta se basa en los siguientes aspectos: a) Virtualización de hardware, b) Para-virtualización, y c) Virtualización a nivel de sistema operativo. La razón de los aspectos a considerar tiene relación con los contenidos programáticos de los espacios académicos del área de infraestructura de TI del Programa Ingeniería de Sistemas y Computación.\\

%Cabe resaltar la actualidad y el impacto de la temática a tratar en el presente trabajo, toda vez que gran parte de las organizaciones hoy en día utilizan la tecnología computacional como un elemento fundamental para potenciar el cumplimiento de sus objetivos \parencite{Pessolani2012}, por lo que se hace necesario contar con el conocimiento preciso acerca de la especificación dinámica de recursos computacionales utilizando tecnologías de virtualización \parencite{Hui2014}. También es importante reconocer y saber gestionar las tecnologías de virtualización que hacen posible soportar el despliegue de recursos tecnológicos en soluciones empresariales de bajo, mediano y alto impacto \parencite{Chiueh2005}. \\

%El trabajo presentado consta de dos partes, la Parte \ref{Parte1}, corresponde a la descripción general de la propuesta presentada ante el Consejo de la Facultad de Ingeniería de la Universidad del Quindío. A su vez, esta parte comprende las siguientes secciones: \ref{justificacion}) Justificación, \ref{Objetivos}) Objetivos y \ref{tematicaADesarrollar}) Temática a desarrollar. Es importante dar claridad al lector que los objetivos y alcance de la propuesta fueron aprobados por el Consejo de la Facultad de Ingeniería, mediante el acta No. 16 del 16 de agosto de 2018. La Parte \ref{Parte2}, corresponde al trabajo de ascenso desarrollado, el cual incluye secciones como las siguientes: \ref{Introduccion}) Introducción, \ref{marcoConceptual}) Marco Conceptual el cual a su vez comprende \ref{fundamentos}) Fundamentos, \ref{historia}) Contexto histórico y \ref{ambitos}) Ámbitos de aplicación, Identificación de herramientas de virtualización y finalmente  \ref{RecursosEducativos}) Recursos educativos presentados como la gestión de un conjunto de herramientas de virtualización.\\



%\alertwarning{Borrador}

Este libro presenta una conceptualización acerca de la infraestructura de Tecnología Informática (TI) y su aporte
hacia entrega de valor en la gestión organizacional.
 En este sentido, se consideran aspectos conceptuales
relacionados con las redes de informáticas, para lo cual se toma como referente al modelo TCP/IP; además, también se incluyen
los conceptos relacionados con los sistemas de cómputo que representan diversas formas del modelo Cliente/Servidor.
Todos estos conceptos se ilustran a través de tecnologías de virtualización.}


\alertwarning{Información base desde el sílabo}

En éste espacio académico, se presentan aspectos conceptuales que le permiten al estudiante comprender los principios y funcionalidades de la administración de la infraestructura de TI, enfocados en estándares y buenas prácticas para mantener la operación de los componentes de TI según las necesidades de las organizaciones cliente. En este sentido, aspectos como la identificación de necesidades organizacionales; el diseño y planeación de soluciones; la implantación y operación; al igual que la monitorización de la Infraestructura de TI, son finalmente elementos centrales para en el perfil profesional de ingeniería de Sistemas y Computación.



%Actualmente, los actores en el desarrollo de la ciencia y tecnología pertenecientes a los
%\textit{\gls{EcosistemaTecnoloogicoDeApoyoALaInvestigacioon}} (\acrshort{ETAI}), particularmente en instituciones de países en vía de desarrollo, presentan dificultades con la visibilidad, el acceso y uso compartido del conjunto de recursos y servicios informáticos. Considerando este problema, la presente propuesta de investigación doctoral plantea especificar un modelo de referencia para la gestión de recursos y servicios informáticos en los \acrshort{ETAI}. A través de este modelo de referencia se pretende brindar lineamientos para propender hacia la solución del problema identificado.\\

%Durante la revisión del estado del arte fue posible identificar múltiples trabajos afines y también fue notable que,
%en el ámbito de los países en vía de desarrollo, aún no se ha logrado consolidar un esquema de solución ampliamente usado para permitir la interoperabilidad del conjunto de recursos y servicios informáticos existentes en los \acrshort{ETAI}. En este sentido, la presente propuesta de investigación toma valor innovador y busca realizarse en beneficio de los \acrshort{ETAI} y en consecuencia fomentar el trabajo voluntario y cooperativo entre diversas instituciones, especialmente aquellas que se encuentran en los países en vía de desarrollo.\\

%Metodológicamente se inicia con la identificación y procesamiento de trabajos académicos con la intención de
%reconocer elementos, procesos y procedimientos necesarios para lograr la visibilidad, el acceso y uso compartido del conjunto de recursos informáticos presentes en los \acrshort{ETAI}. Se continúa concretando la base documental de la investigación, la selección de herramientas tecnológicas y la identificación de necesidades/oportunidades del grupo objetivo. Posteriormente se busca expresar la solución al problema, mediante la especificación del modelo de referencia y la descripción arquitectural del conjunto de artefactos relacionados. Luego se pretende realizar la construcción de un prototipo funcional para instanciar los componentes tecnológicos del modelo de referencia especificado. De forma constante se llevará a cabo la verificación de los artefactos diseñados e implementados que hacen parte del conjunto de elementos de la solución. Finalmente, se identificarán los efectos obtenidos en cada una de las demás etapas en contraste con los objetivos planteados en la investigación.\\

%El modelo de referencia podrá ser instanciado de forma aislada beneficiando a una comunidad particular y
%posteriormente puede ser integrado con otras instancias del modelo de referencia según la necesidad y vocación de cada comunidad científica. Se pretende que el modelo de referencia tenga un enfoque de portal científico, el cual se podrá establecer en las entidades como un elemento con ciclo de vida activo y adaptable, además de incluir los lineamientos necesarios para realizar la gestión de los recursos y servicios informáticos en los \acrshort{ETAI}. Estos lineamientos estarán expresados en diversos niveles de abstracción que van desde los aspectos tecnológicos hasta los organizacionales. De este modo será posible aumentar la visibilidad, el acceso y uso compartido de los recursos y servicios informáticos existentes en los \acrshort{ETAI} sacando provecho de su naturaleza heterogénea.\\


%Se pretende expresar el modelo de referencia en términos de interoperabilidad para los recursos y servicios informáticos. Además, la estrategia de solución parte del supuesto que muchos de los recursos y servicios informáticos existentes cuentan con un esquema de sostenibilidad inherente. Esta iniciativa propende hacia la defensa del acceso a los recursos existentes buscando así la equidad desde el ámbito académico-científico. Del mismo modo, la estrategia fomenta un esquema de cooperación interinstitucional en el que cada institución puede ofrecer de forma voluntaria y cooperativa los recursos y/o servicios informáticos que desee integrar al entorno de cooperación.\\
