\newpage
{\huge \textbf{Abstract}}\\

\alertwarning{Revisar tiempos verbales}


The actors in the development of science and technology belonging to the \acrfull{TERS}, particularly in institutions of developing countries, present difficulties with the visibility, access and shared use of the set of computing resources and services. Considering this problem, the present doctoral research proposal proposes to specify a reference model for the management of computing resources and services in \acrshort{TERS}. Through this reference model, it is intended to provide guidelines for the solution of the identified problem.\\

During the review of the state of the art it was possible to identify multiple related works and it was also noticeable that, in developing countries, it has not yet been possible to consolidate a widely used solution scheme to allow the interoperability of the set of \acrshort{IT} resources and services existing in \acrshort{TERS}. In this sense, the present research proposal takes innovative value and seeks to be carried out for the benefit of \acrshort{TERS} and, consequently, to promote voluntary and cooperative work among different institutions, especially those located in developing countries.\\

Methodologically, it starts with the identification and processing of academic works with the intention of recognizing elements, processes and procedures necessary to achieve visibility, access and shared use of the set of \acrshort{IT} resources present in \acrshort{TERS}. The documentary basis of the research, the selection of technological tools and the identification of needs/opportunities of the target group continue to be specified. Subsequently, the solution to the problem is expressed through the specification of the reference model and the architectural description of the set of related artifacts. Then the construction of a functional prototype is intended to instantiate the technological components of the specified reference model. Verification of the designed and implemented artifacts that are part of the set of elements of the solution will be carried out on an ongoing basis. Finally, the effects obtained in each of the other stages will be identified in contrast with the objectives set out in the research.\\

The reference model can be instantiated in an isolated way benefiting a particular community and later can be integrated with other instances of the reference model according to the need and vocation of each scientific community. It is intended that the reference model will have a scientific portal approach, which can be established in the entities as an element with an active and adaptable life cycle, in addition to including the necessary guidelines for the management of \acrshort{IT} resources and services in the \acrshort{TERS}. These guidelines will be expressed at various levels of abstraction, ranging from technological to organizational aspects. In this way, it will be possible to increase the visibility, access and shared use of the \acrshort{IT} resources and services existing in the \acrshort{TERS}, taking advantage of their heterogeneous nature.\\

%The reference model is intended to be expressed in terms of the organizational, semantic, syntactic and technical dimensions of interoperability for computing resources and services. In addition, the solution strategy assumes that many of the existing \acrshort{IT} resources and services have an inherent sustainability scheme. This initiative is aimed at defending access to public resources, thus seeking the democratization of knowledge and equity from the academic-scientific sphere. Likewise, the strategy promotes an inter-institutional cooperation scheme in which each institution can voluntarily and cooperatively offer the \acrshort{IT} resources and/or services it wishes to integrate into the cooperation environment.\\
