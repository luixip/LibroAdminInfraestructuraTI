%~\vfill
\chapterimage{fondoSecciones} % Table of contents heading image
\thispagestyle{fancy}
\chapter*{Introducción}
\addcontentsline{toc}{chapter}{Introducción}
\markboth{Introducción}{Introducción}

Para la gran mayoría de países es un reto constante fortalecer las capacidades de investigación y producción de conocimiento, al igual que sus sistemas e infraestructura tecnológica, lo anterior para fomentar la participación de diversos actores en la sociedad del conocimiento. Este reto sigue latente en los países en desarrollo y su incumplimiento incrementa las posibilidades de una mayor marginación en comparación con los países desarrollados \parencite{Bautista2019}.

Asimismo, existen diversos actores en la producción de conocimiento a partir de la investigación científica; estos actores están representados entre otros por laboratorios, centros de investigación, instituciones públicas y privadas del sector educativo o industrial con énfasis en el desarrollo e innovación tecnológica. Todos estos actores tienen en común que soportan las actividades para el desarrollo de ciencia y tecnología a través de \acrfull{RSI}. Estos \acrshort{RSI} corresponden a un conjunto heterogéneo que comprende múltiples elementos entre los que destacan los siguientes: datos (por ejemplo, el resultado de experimentos), sistemas especializados (\textit{software} fuertemente acoplado a procesos técnicos), equipos de laboratorio, bases de datos, librerías, sistemas de computación distribuidos y sistemas de computación en la nube, entre otros. Este conjunto de \acrshort{RSI} hace parte de los elementos que sumado a los usuarios y sus relaciones, constituyen en el contexto del presente trabajo, el concepto denominado \acrfull{ETAI}.  Este concepto es extrapolado desde el \textit{ecosistema biológico} y es similar al concepto de \textit{ecosistema tecnológico} para la educación indicado por \textcite{Garcia-Holgado2018}.

Algunos \acrshort{ETAI} están siendo integrados a través de redes nacionales de investigación y educación (\acrshort{NREN} por sus siglas en inglés \textit{\acrlong{NREN}}). Aunque el trabajo realizado por estas \acrshort{NREN} es muy positivo, estas iniciativas aún son insuficientes para brindar los \acrshort{RSI} necesarios en \acrshort{ETAI}, especialmente aquellos ubicados en los países en desarrollo. Esta situación problémica se debe entre otras causas a la carencia de modelos de gobernanza, gestión e interoperabilidad que permitan a los investigadores lograr una adecuada visibilidad y acceso para el uso compartido de los \acrshort{RSI} existentes en las diversas instituciones. Hecho similar es descrito por \textcite{Garcia-Penalvo2018} con respecto a la interoperabilidad de los elementos existentes en los ecosistemas tecnológicos universitarios.

Con respecto a las consideraciones previas sobre esta problemática, a nivel mundial se tiene que la falta de visibilidad, acceso y uso compartido de los \acrshort{RSI} en los \acrshort{ETAI} presentan un claro obstáculo para el desarrollo de proyectos, afectando directamente a grupos de investigación en las universidades y a otros grupos con funciones homólogas, presentes en los diversos actores que generan ciencia y tecnología \parencite{Weigel2020}.

Esta situación también indicada por \parencite{Herrera2009} representa limitaciones en el alcance y tipo de proyectos que pueden ser formulados por estos grupos, lo que en consecuencia podría ralentizar el desarrollo de la ciencia y la tecnología en cada país, trayendo consigo efectos negativos en diversos sectores.

En \acrfull{ALyC} esta problemática representa un obstáculo para la integración regional y limita los procesos de cooperación y crecimiento conjunto \parencite{CINTEL2010, Diaz2016,Moreno-Escobar2007}. Además, la sociedad de la información en \acrshort{ALyC} presenta una compleja problemática social debido a las limitaciones económicas y tecnológicas. Esta situación según lo indica \textcite{Bautista2019} representa un gran desafío para la educación superior y la investigación en un contexto geopolítico donde converge el ámbito local, nacional e internacional.

Particularmente en Colombia el desafío está latente, tal como se evidencia desde el trabajo de \textcite{Castro2009}, donde múltiples aspectos entre ellos el aislamiento en la toma de decisiones y la inequidad presupuestal, conducen a las \acrfull{IES} ya sean públicas o privadas, hacia la adquisición de recursos y servicios informáticos con poca o nula visibilidad, acceso y uso compartido. En este mismo sentido, y según lo señala \textcite{Rico-Bautista2020} se puede evidenciar que las \acrshort{IES} de \acrshort{ALyC} necesitan alineación e integración de la tecnología con los procesos de la organización.

En complemento al problema descrito anteriormente y según el estudio de \textcite{Barbosa2011} citado por \textcite{Garcia-Holgado2018}, los ecosistemas tecnológicos en general presentan limitaciones como las siguientes: a) Establecimiento de las relaciones entre los actores del ecosistema. b) Carencia de la representación adecuada de las personas y sus conocimientos en el modelado de ecosistemas. c) Estabilidad de la interfaz del ecosistema, gestión de la evolución, seguridad y confiabilidad, entre otros. d) Heterogeneidad de licencias de software y evolución de los ecosistemas. e) Barreras técnicas y socio-organizacionales para la coordinación y comunicación de requisitos en proyectos distribuidos geográficamente. f) Infraestructura y herramientas para fomentar la interacción social, la toma de decisiones y el desarrollo entre las organizaciones involucradas.

Además de las limitaciones indicadas anteriormente, se debe considerar que, para los \acrshort{ETAI} se presentan aspectos particulares como: i) Dificultad para que los investigadores puedan acceder a los recursos y servicios informáticos existentes ya sea al interior o exterior de las instituciones. ii) Repetición en la adquisición de recursos y servicios informáticos, en lugar de usar los existentes de forma compartida. iv) Repetición de esfuerzos que corresponde a la puesta a punto de los ambientes especializados de computación, situación que puede desviar a los investigadores de sus objetivos misionales y a los proyectos en aspectos como tiempo y presupuesto. v) Costos adicionales por la adquisición de infraestructura computacional sub-utilizada, lo cual aumenta el costo total de propiedad representado en pólizas, contratos de soporte, mantenimiento, consumo eléctrico, pagos a personal técnico, etc. Esto puede desbordar el presupuesto disponible en instituciones con recursos limitados sin brindar un adecuado retorno de las inversiones. vi) La visibilidad de resultados prevista en los procesos de investigación puede verse afectada por la gestión inadecuada de los recursos y servicios informáticos.

En este contexto, los \acrshort{RSI} como elementos de los \acrshort{ETAI} presentan dificultades para su visibilidad, acceso y uso compartido, limitando el trabajo cooperativo entre los actores que buscan la generación de ciencia y tecnología. De este modo, la expresión del problema tratado en este investagación corresponde a lo siguiente:  "\textit{Los actores en el desarrollo de la ciencia y tecnología pertenecientes a los ETAI, particularmente en instituciones de países en vía de desarrollo, presentan dificultades con la visibilidad, el acceso y uso compartido del conjunto de recursos y servicios informáticos}".

Considerando todo lo anterior, el trabajo realizado en esta investigación, presenta como resultado un \textit{\acrfull{GRSI-ETAI}}. Este modelo establece bases teóricas con un enfoque de portal científico (\acrfull{SG}) para permitir la visibilidad, acceso y uso compartidos del conjunto de \acrshort{RSI} existentes en los \acrshort{ETAI}, especialmente aquellos ubicados en instituciones de países en vía de desarrollo. Particularmente para esta investigación el grupo objetivo estuvo representado por los \acrshort{ETAI} de la \textit{Universidad Tecnológica de Pereira} y la Universidad\textit{ del Quindío}.

Para la realización del modelo \acrshort{GRSI-ETAI} se siguió metodológicamente una adaptación de los trabajos de \textcite{Hernandez-Sampieri2014} mediante las siguientes seis etapas: 1) \textit{Estudio preliminar}, 2) \textit{Análisis}, 3) \textit{Diseño}, 4) \textit{Implementación}, 5) \textit{Validación del diseño y la implementación},  y 6) \textit{Análisis de resultados}. Particularmente, el modelo \acrshort{GRSI-ETAI} considera una línea de base establecida por diversos marcos de referencia para gobierno y gestión de TI con amplia aceptación en la industria informática como, \textit{COBIT 2019} e \textit{ITIL 4}. A partir de estos marcos de referencia, el modelo \acrshort{GRSI-ETAI} se presenta como una extensión de \textit{COBIT 2019}, siendo este último el marco de referencia líder para el gobierno y gestión de la información \parencite{Steuperaert2019}. De este modo, el modelo \acrshort{GRSI-ETAI} se expresa en cuatro componentes como son: a) \textit{Público objetivo}, b) \textit{Principios}, c) \textit{Consideraciones sobre el gobierno de TI} y d) \textit{Nuevo \acrfull{OGG}: Gestionar los \acrshort{RSI} del \acrshort{ETAI}}.









