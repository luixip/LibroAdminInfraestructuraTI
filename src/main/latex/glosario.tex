%% Definción de términos (Glosario)

\newglossaryentry{ciberinfraestructura}
{
	name=Ciberinfraestructura,
	description=
	{
		Concepto conocido en inglés como \textit{Cyberinfrastructure}, el cual según \textcite{Stewart2019} continúa ampliándose y hace referencia a "la integración de supercomputadoras, recursos de datos, visualización y personas que extienden el impacto y la utilidad de las tecnologías de la información"
	}
}


\newglossaryentry{EcosistemaTecnoloogicoDeApoyoALaInvestigacioon}
{
	name={Ecosistema Tecnológico de Apoyo a la Investigación},
	description=
	{
		Este concepto, es acuñado en el presente trabajo y hace referencia a la comunidad investigativa de una institución, considerando sus relaciones y procesos apalancados en el uso de \acrfull{RSI}. Este concepto es extrapolado desde el ecosistema biológico y es similar al concepto de ecosistema tecnológico para la educación indicado por \parencite{Garcia-Holgado2018}
	}
}

\newglossaryentry{computacionEnLaNube}
{
	name=Computación en la nube,
	description=
	{
		Concepto referido en inglés como Cloud Computing y según el \acrfull{NIST} se refiere a “un modelo para permitir el acceso ubicuo, conveniente y por demanda de un conjunto compartido de recursos informáticos configurables (redes, servidores, almacenamiento, aplicaciones y servicios) que se pueden aprovisionar y liberar rápidamente con un mínimo de esfuerzo de gestión o de interacción con el proveedor de servicios” \parencite{Mell2011}
	}
}

\newglossaryentry{Interoperabilidad}
{
	name=Interoperabilidad,
	description=
	{
		Este concepto según \textcite{Lara2007} posee diversas acepciones según el ámbito de aplicación y en el ámbito de la informática se define como "la capacidad del software y del hardware perteneciente a diferentes máquinas y de diferentes marcas comerciales para compartir datos". En ese mismo sentido la Comisión Europea citada en el Libro Blanco de la interoperabilidad de gobierno electrónico para América Latina y el Caribe \parencite{CEPAL2007} la definen como “la habilidad de los sistemas TIC, y de los procesos de negocio que ellas soportan, de intercambiar datos, posibilitar información y conocimiento”; también se indica que la interoperabilidad se desarrolla teniendo presente los aspectos semánticos, organizacionales, técnicos y de gobernanza
	}
}

\newglossaryentry{modelo}
{
	name=modelo,
	description=
	{
		Este término posee diversas acepciones dependiendo del ámbito de aplicación, sin embargo, para el contexto del presente trabajo y según lo indica el diccionario de la lengua española \parencite{RAE_Modelo2020}, el término modelo se define como "Arquetipo o punto de referencia para imitarlo o reproducirlo"
	}
}

\newglossaryentry{PortalCientifico}
{
	name={portal científico},
	description={
		Ver \gls{ScienceGateway}.
	}
}

\newglossaryentry{def:RSI}
{
	name={Recursos y Servicios Informaticos},
	description={Conjunto heterogéneo que comprende múltiples elementos entre los que destacan: datos (por ejemplo, el resultado de experimentos), sistemas especializados (software fuertemente acoplado a procesos técnicos), equipos de laboratorio, bases de datos, librerías, sistemas de computación distribuidos y sistemas de computación en la nube, entre otros
	}	
}


\newglossaryentry{ScienceGateway}
{
	name={Science Gateway},
	description={
		El concepto \acrfull{SG} posee amplia aceptación en países como Estados Unidos y Canadá y es definido según los trabajos de \textcite{Barker2019} y \textcite{Wilkins-Diehr2013} como "\textit{sistemas de información empresarial basados en la web que proporcionan a los científicos acceso fácil y personalizado a las colecciones de datos específicos de la comunidad, herramientas computacionales y servicios de colaboración sobre infraestructuras electrónicas}". El concepto \acrshort{SG} también abarca a otros conceptos tales como "\textit{laboratorios virtuales}" y los "\textit{ambientes virtuales de investigación}", en inglés Virtual Research Enviroment
	}
}


\newglossaryentry{VirtualResearchEnvironment}
{
	name={virtual research environment},
	description={
		Este concepto conocido como \acrfull{VRE} según \textcite{Barker2019} tiene mayor aceptación en Australia y es considerado como una abreviatura de las herramientas y tecnologías que necesita el investigador para realizar su investigación, interactuar con otros investigadores y para hacer uso de los recursos y las infraestructuras técnicas disponibles tanto a nivel local como nacional
	}
}

\newglossaryentry{Virtualizacioon}
{
	name={virtualización},
	description={
		Este término es considerado como la combinación de diferentes tecnologías ofreciendo ventajas como el apoyo para la gestión de grandes volúmenes de datos, la posibilidad de escalar rápidamente y el aprovechamiento de grandes capacidades de cómputo; técnicamente y según se señala en los estudios de \textcite{Kusnetzky2011}  y \textcite{AbdElRahem2016}, la virtualización permite abstraer aplicaciones y los componentes subyacentes de hardware para ser presentados como una vista lógica o virtual de estos recursos. Esta vista lógica puede ser en ocasiones notablemente diferente a la vista física, la cual por lo general se construye a partir del exceso de recursos de computación, tales como el poder de procesamiento, la memoria, la capacidad de almacenamiento o incluso el ancho de banda \parencite{Stallings2015}
	}
}

